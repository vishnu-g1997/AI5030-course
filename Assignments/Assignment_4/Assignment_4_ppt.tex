\documentclass[10pt]{beamer}

\usetheme[progressbar=frametitle]{metropolis}
\usepackage{appendixnumberbeamer}

\usepackage{booktabs}
\usepackage[scale=2]{ccicons}
\usepackage{tikz,lipsum,lmodern}
\usepackage[most]{tcolorbox}
\usepackage{pgfplots}
\usepgfplotslibrary{dateplot}

\usepackage{xspace}
\newcommand{\themename}{\textbf{\textsc{metropolis}}\xspace}

\title{Assignment 4}
\subtitle{Problem 49 UGC Math June-2018}
% \date{\today}
\date{}
\author{Vishnu G}
\institute{Indian Institute of Technology Hyderabad}
% \titlegraphic{\hfill\includegraphics[height=1.5cm]{logo.pdf}}
\begin{document}
\maketitle

\begin{frame}[fragile]{Question}
49. A standard fair die is rolled until some face other than 5 or 6 turns up. Let X denote the face value of the last roll, and A = [X is even] and B = [X is at most 2]. Then.\\
1. $P(A \bigcap B) = 0$ \\
2. $P(A \bigcap B) = 1/6$ \\
3. $P(A \bigcap B) = 1/4$ \\ 
4. $P(A \bigcap B) = 1/3$ \\

Answer: 3
\end{frame}
\begin{frame}{Solution}
\begin{enumerate}
    \item $P(X=x,Y=y)$ = $P(X=x/Y=y) \times P(Y=y)$
    \item $P(X=x)$ = $\sum_{Y_j\in R} P(X=x) P(Y=y_j)$  
\end{enumerate}
\end{frame}

\begin{frame}{Solution}
\begin{enumerate}
    \item Given X is the face value of the dice\\
\item 
\begin{align*}
&  P[(X \in (2,4,6), X \in (1,2))/T] \\
&= P[(X = 2)/T] \\
&= \frac{1}{6} 
\end{align*}
\end{enumerate}
\end{frame}

\begin{frame}{contd..}
Joint Probability of $X=2$ and T (i.e Trail is allowed )
\begin{align*}
P(X=2,T_1) &= P(X=2/T_1)  P(T_1) = \frac{1}{6} \times 1 \\
P(X=2,T_2) &= P(X=2/T_2)  P(T_2) \\
           &=\frac{1}{6} \times P(X_1 \in \{5,6\} ) = \frac{1}{6} \times \frac{1}{3}\\
P(X=2,T_i) &= P(X=2/T_i) P(T_i) = \frac{1}{6} \times \frac{1}{3^{i-1}}
\end{align*}
Probability that trail is continued till $i^{th}$ time 
\begin{equation*}
    P(T_i) = \Pi_{i=1}^{i-1} P(X_i \in \{5,6\})= \frac{1}{3^{i-1}}
\end{equation*}
\end{frame}

\begin{frame}{Contd..}
Marginal probability of P(X=2)
\begin{align*}
    P(X=2) &= \sum_{i=1}^\infty P(X=2,T_i) = \sum_{i=1}^\infty  P(X=2/T_i) \times P(T_i) \\
       &= \frac{1}{6} \times (1+\frac{1}{3}+\frac{1}{3^2} + ....+\frac{1}{3^{i-1}} + .... + \infty)\\
        &= \frac{1}{6} \times \frac{1}{1-\frac{1}{3}}\\
        &= \frac{1}{6} \times \frac{3}{2} \\
        &= \frac{1}{4}
\end{align*}
\end{frame}
\end{document}
