\documentclass[10pt]{beamer}

\usetheme[progressbar=frametitle]{metropolis}
\usepackage{appendixnumberbeamer}

\usepackage{booktabs}
\usepackage[scale=2]{ccicons}
\usepackage{tikz,lipsum,lmodern}
\usepackage[most]{tcolorbox}
\usepackage{pgfplots}
\usepgfplotslibrary{dateplot}

\usepackage{xspace}
\newcommand{\themename}{\textbf{\textsc{metropolis}}\xspace}

%\{AI5030: Probability and Stochastic Process}"""
\title{Traffic Flow Prediction at signalised road intersections Using Markov chain and ANN. A comparison}

% \date{\today}
\date{}
\author{\textbf{Vishnu G} :  \textbf{AI22MTECH02001}}
\institute{Indian Institute of Technology Hyderabad}
% \titlegraphic{\hfill\includegraphics[height=1.5cm]{logo.pdf}}
\begin{document}
\maketitle

\begin{frame}{Contents}
\title{}
\begin{itemize}
 \item Motivation
 \item Markov Chain Model of Traffic congestion prediction
 \item ANN for Traffic congestion prediction
 \item Comparison
 \item Conclusion
 \item Reference
\end{itemize}
\end{frame}

\begin{frame}{Motivation}
\begin{itemize}
    \item Traffic congestion is a prevailing problem globally
    \item Predicting Traffic congestion is necessary for traffic modeling, better planning and road traffic control 
    \item Markov chain models are heuristic sequential models, which find the next traffic state based on present state
    \item Artificial Neural networks like RNNs and LSTMs can store continuous sequential information of all past states to predict the next state
\end{itemize}

\end{frame}

\begin{frame}{Markov Chain for Traffic congestion prediction}
\begin{itemize}
\item States 
\begin{itemize}
\item Low Traffic (State - a)
\item Moderate Traffic ( State - b)
\item High Traffic (State - c)
\end{itemize}
\item{State Transition Matrix}
\begin{align*}
\begin{bmatrix}
    P_{a,a}  &  P_{a,b} &  P_{a,c} \\
    P_{b,a}  &  P_{b,b} &  P_{b,c} \\
    P_{c,a}  &  P_{c,b} &  P_{c,c}
\end{bmatrix}
\end{align*}
\end{itemize}

\end{frame}

\begin{frame}
\begin{itemize}
\item This problem is solved using assumption of First order Markov chain, where next state 
depends only the present 
\item{Steady State probabilities}
\begin{align*}
\begin{bmatrix} e1 & e2 & e3 \end{bmatrix} 
= 
\begin{bmatrix} e1 & e2 & e3 \end{bmatrix} \begin{bmatrix}
    P_{a,a}  &  P_{a,b} &  P_{a,c} \\
    P_{b,a}  &  P_{b,b} &  P_{b,c} \\
    P_{c,a}  &  P_{c,b} &  P_{c,c}
\end{bmatrix}
\end{align*}
\end{itemize}
\end{frame}

\begin{frame}{Contd..}
\begin{itemize} 
\item  Traffic is observed on hourly basis and one transition matrix each is maintained of each weekday
\item Transition matrices are updated after observation are made
\item Process of moving from one state to another state can be derived from the equation
\begin{align*}
    P(K) = P(0) \times P_{i,j}^k
\end{align*}
Where,
\begin{itemize}
\item P(0) is Traffic volume for a low, moderate, and high vehicle movement
\item $P_{i,j}$ is state transition Probability matrix
\item P(K) Probability of at a particular hour on a particular day of the week
\end{itemize}
\end{itemize}
\end{frame}

\begin{frame}{Contd..}
\begin{itemize}
\item Probabilities for every next hour for every possible state transition is calculated from 
the state transition matrix
\item Present state is identified after observing the density of incoming vehicles
\item Next state is predicted as whichever state transition has highest probability.
\item Transition matrix is updated after every observation 
\end{itemize}
\end{frame}

\begin{frame}{ANN for Traffic congestion prediction}
Data sets were prepared after observing speed, distance of vehicles 
and then output observed was in terms of time taken for vehicles to cross the signal.
\begin{itemize}
    \item Inputs : Speed, Distance
    \item Outputs: Time taken to cross the signal in the intersection
\end{itemize}
\end{frame}

\begin{frame}{Contd..}
\begin{itemize}
    \item Neural network architecture and training conditions
    \begin{itemize}
    \item Number of hidden layers : 1
    \item Number of neurons in hidden layer : 9
    \item Number of epochs the NN is trained 100
    \item Best results on test data observed at epoch 7
    \end{itemize}
    \item After time needed for each vehicle to cross signal is calculated, we can find if traffic will be congested at every continuous intervals.
\end{itemize}
\end{frame}

\begin{frame}{Comparison and Conclusions}
\begin{itemize}
    \item According to the paper ANN showed better performance when compared to heuristic model like Markov chains
    \item Although Markov chains show slightly lesser performance, computation power required is quite less when compared to ANN.
    \item Markov chains are far less complex when compared to neural networks but give comparable and considerable performance
    
\end{itemize}

\end{frame}

\begin{frame}{Reference}
    Title: Traffic flow Prediction at Signalized Road Intersections: 
    A case of Markov Chain and Artificial Neural Network Model
    Link: https://ieeexplore.ieee.org/document/9476173 \\
    Published in : 2021 IEEE 12th International Conference on Mechanical and Intelligent Manufacturing Technologies
    
\end{frame}
\end{document}
