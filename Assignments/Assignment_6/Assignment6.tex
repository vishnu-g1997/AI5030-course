\documentclass[10pt]{beamer}

\usetheme[progressbar=frametitle]{metropolis}
\usepackage{appendixnumberbeamer}

\usepackage{booktabs}
\usepackage[scale=2]{ccicons}
\usepackage{tikz,lipsum,lmodern}
\usepackage[most]{tcolorbox}
\usepackage{pgfplots}
\usepgfplotslibrary{dateplot}

\usepackage{xspace}
\newcommand{\themename}{\textbf{\textsc{metropolis}}\xspace}

\title{Assignment 6}
\subtitle{Problem 119 UGC Math Dec-2017}
% \date{\today}
\date{}
\author{Vishnu G}
\institute{Indian Institute of Technology Hyderabad}
% \titlegraphic{\hfill\includegraphics[height=1.5cm]{logo.pdf}}
\begin{document}
\maketitle

\begin{frame}[fragile]{Question}
119.) Arrival of customers in a shop is a Poisson
process with intensity $\lambda = 2$. Let X be the number of customers entering during the time interval (1, 2) and let Y be the number of customers entering during the time interval (5,10). Which of the following are true? \\
\begin{enumerate}
    \item $P(X = 0 | (X+Y=12)) = ({\frac{5}{6}})^{12}$\\
    \item $X$ and $Y$ are in-dependant\\
    \item $X+Y$ is a Poisson with parameter 6\\
    \item $X-Y$ is a Poisson with parameter 8
\end{enumerate}

Answer: 1,2
\end{frame}
\begin{frame}{Definitions}
\begin{enumerate}
\item Poisson distribution is limiting Bernoulli distribution Bern(n,p), where $n -> \infty, p->0$
\item Since all Bernoulli trails are in-dependant RVs in each disjoint intervals are also disjoint
\item MGF of Poisson distribution is given by 
\begin{equation}
    M_{X}(t) = e^{(e^t-1)\lambda}
\end{equation}
\end{enumerate}

\end{frame}

\begin{frame}{Poisson as limiting Bernoulli Distribution }
\begin{align*}
    X &\sim{Bin(n,p)}, n \to \infty, p \to 0, \lambda = np\\
    P(X=k) &= \binom nk (p^k)(1-p)^{(n-k)} \\
           &= \frac{n!}{(n-k)! k!} \frac{\lambda^k}{n^k} (1 - \frac{\lambda}{k})^{-k} (1 - \frac{\lambda}{n})^n \\
           &= \frac{\lambda^k}{k!} e^{-\lambda}
\end{align*}
\end{frame}

\begin{frame}{MGF of Poisson}
\begin{align*}
    P(X=k) &= \frac{\lambda^k e^{\lambda}}{k!}\\
    M_x(t) &= E(e^{tX}) \\
    &= \sum_{n=0}^{\infty} \frac{\lambda^n e^{-\lambda}}{n!} e^{tn}\\
    &= e^{-\lambda} \sum_{n=0}^{\infty} \frac{(\lambda e^{t})^n }{n!} \\
    &= e^{(e^t-1)\lambda}
\end{align*}
\end{frame}
\begin{frame}{MGF of two Independent Poisson distributions}
$M_X(t) &= E(e^{tX})$, $M_Y(t) &= E(e^{tY})$ \\
\begin{align*}
     M_{X+Y}(t) &= E(e^{t(X+Y)})\\
     &= E(e^{tX}) E(e^{tY}) \\
     &= e^{(e^t-1)\lambda_1} e^{(e^t-1)\lambda_2}\\
     &= e^{e^t (\lambda_1 + \lambda_2) - (\lambda_1 + \lambda_2)}
\end{align*}
\end{frame}
\begin{frame}{Option 1}
$P(X = 0 | (X+Y=12)) = ({\frac{5}{6}})^{12}$4
\begin{align*}
    P(X=0|(X+Y=12)) &= P(X = 0, X+Y=12) / P(X+Y=12) \\
    &= P(X = 0, Y = 12) / P(X+Y=12) \\
    &= P(X = 0) P(Y=12) / P(X+Y=12) \\
    &= \frac{e^{-2} 2^0} {0!} \frac{e^{-2(5)}5^{12}}{12!} \times (10)^{12} \times \frac{12!}{(2+10)^{12} e^{-12}}\\
    &= (\frac{10}{12})^{12}\\
    &= (\frac{5}{6})^{12}
\end{align*}
\end{frame}

\begin{frame}{Option 2}
X, Y are Poisson distributions of a disjoint intervals so they are in-dependant
\end{frame}

\begin{frame}{Option 3}
From the MGF of two Poisson distributions we know that X+Y is Poisson and the rate is $\lambda_1+\lambda_2 = 2+10 = 12$\\
So Option 3 is incorrect
\end{frame}

\begin{frame}{Option 4}
From the MGF of two Poisson distributions we know that X+Y is Poisson and the rate is $\lambda_1+\lambda_2 = 2-10 = -8$\\
So Option 4 is incorrect
\end{frame}
\end{document}
