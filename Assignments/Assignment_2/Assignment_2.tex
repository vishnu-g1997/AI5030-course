\documentclass{article}
\usepackage[utf8]{inputenc}
\usepackage{setspace}
\usepackage{gensymb}
\singlespacing
\usepackage[cmex10]{amsmath}
\usepackage{caption}

\usepackage{amsthm}

\usepackage{mathrsfs}
\usepackage{txfonts}
\usepackage{stfloats}
\usepackage{bm}
\usepackage{cite}
\usepackage{cases}
\usepackage{subfig}

\usepackage{longtable}
\usepackage{multirow}
\usepackage{fancyhdr}
\usepackage{enumitem}
\usepackage{mathtools}
\usepackage{steinmetz}
\usepackage{tikz}
\usepackage{circuitikz}
\usepackage{verbatim}
\usepackage{tfrupee}
\usepackage[breaklinks=true]{hyperref}
\usepackage{graphicx}
\usepackage{tkz-euclide}
\title{Assignment2}
\author{Vishnu Gollamudi - AI22MTECH02001}
\pagestyle{fancy}
\fancyhf{}
\rfoot{https://github.com/vishnu-g1997/AI5030-course }
\begin{document}
\huge \textbf{Assignment2}
\section{Question 58 - 2019 UGC}
A sample of size $n = 2$ is drawn from a population of size $N = 4$ using probability proportional to size
without replacement sampling scheme, where the probabilities proportional to size are\\
The probability of inclusion of unit 1 in the sample is
\begin{center}
\begin{tabular}{ c c c c c}
 i  & 1   & 2   & 3   & 4   \\ 
 Pi & 0.4 & 0.2 & 0.2 & 0.2 \\  
\end{tabular}
\end{center}
\begin{enumerate}
    \item 0.4 
    \item 0.6
    \item 0.7
    \item 0.75
\end{enumerate}
\large \textbf{Answer : Option - 3 - (0.7)}\\
\section{Definition of probability proportional to size without replacement:}
In a simple random sampling scheme, every unit in the population, has an equal probability of getting selected for a sample subset. But when if in a given population sizes of different classes are different, then probability that a sample belonging to a certain class ( or unit ) differs. Sampling is needed when working with large set of population is difficult to handle, and we reduce it and also ensure that all the classes or units have proportional representation. For example lets say if there are 4 classes in a population and class 1 has 40 $\%$ of representation then after sampling the probability of class 1's representation is 40$\%$. There are various schemes for doing the same. Here we discuss PPS without replacement.\par
In a probability scheme without replacement, when the initial probabilities of selection are unequal, then the probability of drawing a specified unit of the population at a given draw changes with the draw. Also immediate repetition of a sample from same class is not allowed i.e, if a sample from class 1 is selected in the first draw, then it is not allowed to draw a sample from same class in the subsequent draw. \par
\section{Solution:}
If $S_1$ is from unit 1 ( or class 1), then the probability that $S_1$ is drawn in the first draw is $P_i$ = $P_1$, which is already given as 0.4.\\
Now, $P_{i(2)}$ which is probability that $i^{th}$ unit is drawn from the population in the second draw. It can be derived as follows:\\
$P_{i(2)}$ can occur in following possible ways:\\
\begin{itemize}
    \item $S_1$ is selected in $1^{st}$ draw and $S_i$ is selected at 2 draw
    \item $S_2$ is selected in $1^{st}$ draw and $S_i$ is selected at 2 draw
    \item .
    \item .
    \item $S_{i-1}$ is selected in $1^{st}$ draw and $S_i$ is selected at 2 draw
    \item $S_{i-1}$ is selected in $1^{st}$ draw and $S_i$ is selected at 2 draw
    \item $S_{N}$ is selected in $1^{st}$ draw and $S_i$ is selected at 2 draw
\end{itemize}
It has to be noted that $S_i$ is being skipped in for first draw because repition is not allowed in PPS without replacement scheme.\\
$S_1$ is selected in $1^{st}$ draw and $S_i$ is selected at 2 draw is derived as\\
Let $S_1$ is selected in $1^{st}$ draw be Event A, $S_i$ is selected at 2 as Event B
\begin{align*}
P (A,B) &= P(A) \times P(B/A), 
\quad \textit{where}, P(A) = P_1, \\
P(B) &=\frac{P_i}{1-(P_1)}
\end{align*}
Population of $P_i$ over total remaining population, which is $1 - P_1$, because $P_1$ can not be selected again\\
It should be noted that dependency of event B on A is removed inherently when, $S_i$ is skipped in the first draw when calculating the entire probability.
\begin{align*}
P_{i(2)} &= P1 \times \frac{P_i}{1-P_1} + P2 \times \frac{P_i}{1-P_2} + ... \\ &+ P_{i-1} \times \frac{P_{i}}{1-P_{i-1}} + P_{i+1} \times \frac{P_{i}}{1-P_{i+1}} + P_{N} \times \frac{P_{i}}{1-P_{N}}\\
P_{i(2} &= \sum_{j=i}^N P_j \times \frac{P_i}{1-P_j} - P_i \times \frac{P_i}{1-P_i}
\end{align*}
Now, in the given problem, it is asked the inclusion probability of $1^st$ unit. Which means, the probability that $1^st$ unit is drawn in the $1^st$ draw or that is drawn in the second draw. Since in the derivations we removed dependency inherently, it is equal sum of those two probabilities(we subtracted the probability that $1^st$ unit is drawn in the $1^st$ draw \textbf{and} is drawn in the second draw which is $P_i \times \frac{P_i}{1-P_i})$.\par
Now $P_1$ is given as 0.4. Need to find $P_{1(2)}$.
\begin{align*}
    P_{i(2)} &= \sum_{j=i}^N P_j \times \frac{P_i}{1-P_j} - P_i \times \frac{P_i}{1-P_i}\\
    &= 0.4 \times \left (\frac{0.2}{1-0.2} + \frac{0.2}{1-0.2} + \frac{0.2}{1-0.2} + \frac{0.4}{1-0.4} - \frac{0.4}{1-0.4} \right)\\
    &= 0.4 \times \left (\frac{0.6}{0.8} \right)\\
    &=0.3
\end{align*}
Hence The probability of inclusion of unit 1 in the sample is $0.4+0.3 = 0.7$
\end{document}
